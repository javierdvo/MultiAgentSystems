\section{Answers}
In this sections the answers for homework one are given.
\paragraph{a}
\label{Types_Of_Interaction}
		\textbf{Cooperatively:} Solving a task together, e.g. Homework in a group.
		\textbf{Neutrally:} Solving non-competitive task independently, e.g. Driving a car inner city.
		\textbf{Competitively:}Solving opposing task, e.g. Football (soccer).
\paragraph{b}
\label{Rational_Autonomous_Agent}
	 A \textbf{rational} agents optimizes it's performance measure. A \textbf{autonomously} agent uses it's own perception to regulate itself.
\paragraph{c}
\label{Interaction}
	The agent acts upon the world state with a given sets $A_i$ of \textbf{actions}. Each action is assigned with an \textbf{utility} $u_i$.
	Interaction means \textit{communication}. An agents communication can be quantified via the spoken \textbf{language} and \textbf{protocols}.
\paragraph{d}
\label{Relfex_policies}
	A normal policy includes the observation \textit{history} $\pi(o_{1:t})= a_t $.
	A \textbf{reflex policy} takes only the current observation into account $\pi(o_t)=a_t$.
\paragraph{e}
\label{Markov Property}	
	The \textbf{markov property} uses the baysian property to explain, that the current state of an environment and its participants, also describes it's action and state \textit{history} $p( s_{t+1}|s_{a:t} , a_{1:t} ) = p(s_{t+1}|s_t,a_t)$ .
\paragraph{f}
\label{Observability}
	An \textbf{observable world} provides the agent with \textit{complete information} about the world $o_t = s_t $.
	A \textbf{partially observable} world provides the agent with information drawn from a distribution $o_t \sim p(o_t|s_t)$.
\paragraph{g}
\label{Utility function}
	The utility function is "a mapping from states of the world to real numbers, indicating the agent's level of happiness with that state of the world ".
	\textit{From: 01-Introduction slide 24}
\paragraph{h}
\label{Best Response Operator}
	The \textbf{best response operator} $B_i$ for player $i$ chooses the action $a_i$, that maximizes the players $i$ utility $u_i$, given the knowledge of how every other players will act $a_{-i}$ .
\paragraph{i}
\label{Nash Equilibria}
	...or correlated \textit{Nash equilibria}, Yes. (At least in the definition of game theory, we know so far.)
\paragraph{j}
\label{Zero-sum games}
	\begin{enumerate}
		\item[Zero-sum.] The sum of the players utilities  for any single time step $t$ is zero:  $\sum_i(u) = 0$.
		\item[Competition.] Since the utilities sum of the agents actions is zero, the agents are in competition to each other.
		\item[Nash equilibria.] Since there has to be a \textit{looser} for any time step t, the Nash equilibrium has to be a \textit{mixed Nash equilibrium}.
		
	\end{enumerate}

%\subsection{Problem 1.2 -Answers}
%\label{Answers for 1.2}
%\paragraph{a}
%\label{Prisoners_Dilemma}
%\begin{enumerate}
%	\item[1] The Pay-Off table describes the prisoners dilemma game.
%	\item[2] The utility order for the stag hunt game is: $u_1>u_3=u_4>u_2$.
%	\item[3] The utility order for the hawk-dove game is: $u_3>u_1>u_2>u_4$.
%	\item[4] The \textit{stag-hunt game} is a coordination game. Probably because, there exists a utility for an equal action, that dominates all other action combinations.
%	\item[5] 
%\end{enumerate}
%\paragraph{a}
%\label{Pareto_Optimality}
%
