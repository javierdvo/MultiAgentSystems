\section{Problem 2.1 Theoretical Questions}
\label{2.1}

\paragraph{a.1}
A normal game has no stages. An extensive game consists of a sequential decisions.
\paragraph{a.2}
Yes, but it's NP-hard.
\paragraph{a.3}
The Pros and Cons to express a game in extensive form are:
\begin{enumerate}
	\item {Pros} The subgames are easy to identify. And thus a NE might be easier to find. If the game is defined by a sequence of action, the game is easier to formulate.
	\item {Cons:} Pareto-dominated actions are easier to see in a normal game form.
\end{enumerate}
\paragraph{a.4}
Yes. %04.18
\paragraph{a.5}
A threat is a non-optimal choice of an agents openent in the next stage. This choice reduces both agents utility.
\paragraph{a.6}
%see slide 4.26 ff i guess
todo:inline answer
\\
\\

\paragraph{b.1}
%todo: include tree grap
Let $\Gamma$ be an extensive game with perfect information*, with player function $P$. For any nonterminal history* $h$ of $\Gamma$, the subgame $\Gamma(h)$ following the history $h$ is the following extensive game:
\begin{enumerate}
\item \textit{Players} The players in $\Gamma$
\item \textit{Terminal histories} The set of all sequences $h'$ of actions such that $(h,h')$ is a terminal history of $\Gamma$
\item \textit{Player function} The player function $P(h,h')$ is assigned to each proper sub-history $h'$ of a terminal history
\item \textit{Preferences} Each player prefers $h'$ to $h''$ if and only if she prefers $(h,h')$ to $(h,h'')$ in $\Gamma$
\end{enumerate}
From 04-ExtensiveForm p.22. 

*1: \textit{Perfect information:} Every player has all the information from the given extensive-form tree (c.f. 04-ExtensiveForm p. 10).
*2: \textit{nonterminal history:} "Sequence of actions taken by the players up to some decision point[.]", that does not reach until a payoff distribution (c.f. 04-ExtensiveForm. p.5).

\paragraph{b.2}
\label{def:SPE}
In a \textit{sub-game perfect equilibrium} each player's strategy is required to be optimal, given the other players' strategies, not only at the start but at every possible history. From 04-ExtensiveForm p.21.

The strategy profile $s*$ in an extensive game with perfect information is a \textit{sub-game perfect equilibrium (SPE)}, if for every player $i$ and every history $h$ after which it is player $i's$ turn to move,
$u_i(O_h(s*)) \geq u_i(O_h(r_i,s_{-i}*))$ for every strategy $r_i$ of player $i$,
where $u_i$ is a payoff function that represents the player $i's$ preferences and $O_h(s)$ is the terminal history consisting of $h$ followed by the sequence of actions generated by $s$ after $h$.
From 04-ExtensiveForm p.26

\paragraph{b.3}
Every SPE is a NE because,
"$u_i(O_h(s*)) \geq u_i(O_h(r_i,s_{-i}*))$ for every strategy $r_i$ of player $i$,"
is excatly the best response operator in a \textit{normal game}.
%todo: validate if anser is enough

\paragraph{b.4}
It is possible for a an extensive form game to have a set $s*$ of SPE's  (c.f. \ref{def:SPE}).
Example: $u_i(O_h(s*)) = u_i(O_h(r_i,s_{-i}*))$ for every strategy $r_i$ of player $i$.


\paragraph{c.1}
\label{Problem_definition_baysian_games}
Baysian games deal with imperfect information.
An agents perceives chances of an other agents utilities (Player twos has assigned chances to play different version of the game.)
From 04-ExtensiveForm p. 61 ff.

\paragraph{c.2}
\label{baysian_to_normal_games_conversion}
"We can [...] convert a Bayesian game to a game in normal form over the set of pure strategies of the players."
(from: 04-ExtensiveForm games p. 66)
Meaning: Compute the expected utilies and compute the Nash Equilibria.

\paragraph{c.3}
\label{Expected_utilities}
\begin{enumerate}
	\item {Ex-ante:} the agent knows nothing about anyone's actual type
	\item {ex-interim:} an agent knows his own type but not the types of the other agents
	\item {ex-post:} the agent knows all agents' types
\end{enumerate}
From 04-ExtensiveForm games p.66
\\
\textit{Ex-ante} is the sum of all possible stratety-type-expectations. 
\textit{Ex-interim} is the sum over the oponents stratety-type-expectations given my preferences/types.
\textit{Ex-post} is the sum over all expected strategies
